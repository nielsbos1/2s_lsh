\begin{abstract}
With the Internet becoming more intertwined with every aspect of our life, there is an exponentially growing amount of unstructured information available. To support the customer in product selection, websites have popped up that aggregate product listing data from other websites and point customers to the best deal out there. The \textit{multi-component similarity method with preselection+} has been developed to determine whether products from different Web shops relate to the same real world entity. This method uses a \textit{locality-sensitive hashing} (LSH) to achieve a large reduction in candidate comparisons at a relatively small loss in recall. This paper investigates how to further optimize the LSH step in this application. We run an experimental bootstrapped evaluation on a dataset of television products extracted from four different Web shops. The current sketching scheme in use in this application is the \textit{min-wise independent permutations locality-sensitive hashing} (MinHash) scheme. We implement the sketching scheme \textit{fast similarity sketching} (FSS), and find that its effectiveness is on at least equivalent to MinHash and often superior. As FSS is also more efficient, it should be considered as a superior choice of Jaccard similarity sketching scheme. Furthermore, we propose a parametrized framework for the amplification of LSH schemes. By optimizing the parameters of the amplified LSH scheme for a wide range of configurations, we provide insight into the conditions under which amplification adds value. The bootstrap evaluation confirms that amplification can achieve a substantial reduction in candidate comparisons, while keeping the recall on the same level. 
\end{abstract}