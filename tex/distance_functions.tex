\section{Distance functions}
Items are mapped to a space.  A distance measure on a space is a function $d(x,y)$ that takes two points in the space as arguments and produces a real number. The following conditions apply:
\begin{enumerate}
    \item $d(x,y) \geq 0 $ (no negative distances)
    \item $d(x,y) = 0$ if and only if $x = y$ (distances are always positive, unless it concerns perfectly similar points)
    \item $(d(x,y) = d(y,x) $ (distance is always symmetric)
    \item $d(x,y) \leq d(x,z) + d(z,y)$ (triangle inequality)
    
\end{enumerate}

Triangle-inequality axiom results in distance always describing the shortest path between two arbitrary points. 

\subsection{Types of distances}
\subsubsection{Euclidean distance}
The euclidean distance would be what comes to mind when we normally think of a "distance". 
\newline
Definition of \textit{n-dimensional Euclidean space}:
\begin{center}

Space which consists of points that are vectors of \textit{n} real numbers
\end{center}

The most commonly-used distance measure is the $L_{2}-norm$:
\begin{equation}
d\left(\left[x_{1}, x_{2}, \ldots, x_{n}\right],\left[y_{1}, y_{2}, \ldots, y_{n}\right]\right)=\sqrt{\sum_{i=1}^{n}\left(x_{i}-y_{i}\right)^{2}}
\end{equation}
To any inner product space (which in itself is a generalization of the Euclidean space) the Cauchy-Schwarz inequality applies. 
\begin{equation}
|\langle\mathbf{u}, \mathbf{v}\rangle| \leq\|\mathbf{u}\|\|\mathbf{v}\|
\end{equation}

Proof of triangle equality: 
\begin{equation}
\begin{aligned}
\|\mathbf{u}+\mathbf{v}\|^{2} &=\langle\mathbf{u}+\mathbf{v}, \mathbf{u}+\mathbf{v}\rangle \\
&=\|\mathbf{u}\|^{2}+\langle\mathbf{u}, \mathbf{v}\rangle+\langle\mathbf{v}, \mathbf{u}\rangle+\|\mathbf{v}\|^{2} \quad \text { where }\langle\mathbf{v}, \mathbf{u}\rangle=\overline{\langle\mathbf{u}, \mathbf{v}\rangle} \\
&=\|\mathbf{u}\|^{2}+2 \operatorname{Re}\langle\mathbf{u}, \mathbf{v}\rangle+\|\mathbf{v}\|^{2} \\
& \leq\|\mathbf{u}\|^{2}+2|\langle\mathbf{u}, \mathbf{v}\rangle|+\|\mathbf{v}\|^{2} \\
& \leq\|\mathbf{u}\|^{2}+2\|\mathbf{u}\|\|\mathbf{v}\|+\|\mathbf{v}\|^{2} \\
&=(\|\mathbf{u}\|+\|\mathbf{v}\|)^{2}
\end{aligned}
\end{equation}
Then, taking square roots, we get 

\begin{equation}
\|\mathbf{u}+\mathbf{v}\| \leq\|\mathbf{u}\|+\|\mathbf{v}\|
\end{equation}

The $L_{2}-norm$ can be generalized into the $L_{r}-norm$, which is defined as 
\begin{equation}
d\left(\left[x_{1}, x_{2}, \ldots, x_{n}\right],\left[y_{1}, y_{2}, \ldots, y_{n}\right]\right)=\left(\sum_{i=1}^{n}\left|x_{i}-y_{i}\right|^{r}\right)^{1 / r}
\end{equation}
The case of $r=2$ refers to the Euclidean distance, while the case of $r=1$ refers to the $Manhattan distance$.
As r approaches $\infty$, only the dimension with the largest difference matters. Hence, the $L_{\infty}-norm$ is defined as the maximum of $|x_i-y_i|$ over all dimensions $i$

\subsubsection{Jaccard distance}
The Jaccard distance is defined as $1 - SIM(x,y)$, where $SIM(x,y)$ is defined as the Jaccard similarity between two sets. The Jaccard similarity is the ratio of the intersection and union of sets $x$ and $y$. 